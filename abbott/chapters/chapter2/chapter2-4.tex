\section{The Monotone Convergence Theorem and a First Look at Infinite Series}

\begin{exercise}
  \enum{
  \item Prove that the sequence defined by $x_{1}=3$ and
    $$
    x_{n+1}=\frac{1}{4-x_{n}}
    $$
    converges.
  \item Now that we know $\lim x_{n}$ exists, explain why $\lim x_{n+1}$ must also exist and equal the same value.
  \item Take the limit of each side of the recursive equation in part (a) to explicitly compute $\lim x_{n}$.
  }
\end{exercise}

\begin{solution}
  \enum{
  \item $x_n$ is monotone decreasing; and because $x_n < 4$, $x_n$ is always greater than zero, which means that it is bounded and therefore convergent.
  \item The sequences $x_n$ and $x_{n+1}$ are just the same sequence, only offset by one term.
  \item 
  $$
  \begin{aligned}
  \lim x_{n+1} = \lim \left( \frac{1}{4-x_n}\right)
  &\implies s = \frac{1}{4-s} \\
  &\implies s^2 - 4s +1 = 0 \\
  &\implies s = 2\pm\sqrt 3
  \end{aligned}
  $$
  However, $s$ cannot be $2+\sqrt 3$ as $2 + \sqrt 3 > 3$, so $s = 2-\sqrt 3$.
  }
\end{solution}

\begin{exercise}
  \enum{
  \item Consider the recursively defined sequence $y_{1}=1$
  $$
  y_{n+1}=3-y_{n}
  $$
  and set $y=\lim y_{n} .$ Because $\left(y_{n}\right)$ and $\left(y_{n+1}\right)$ have the same limit, taking the limit across the recursive equation gives $y=3-y$. Solving for $y$, we conclude $\lim y_{n}=3 / 2$.
  What is wrong with this argument?
  \item This time set $y_{1}=1$ and $y_{n+1}=3-\frac{1}{y_{n}}$. Can the strategy in (a) be applied to compute the limit of this sequence?
  }
\end{exercise}

\begin{solution}
  \enum{
  \item We don't know that it converges at all. In the previous exercise, we showed that the sequence was convergent before taking the limit.
  \item Yes, because it is bounded above and monotone increasing.
  }
\end{solution}

\begin{exercise}
  \enum{
  \item Show that
    $$
    \sqrt{2}, \sqrt{2+\sqrt{2}}, \sqrt{2+\sqrt{2+\sqrt{2}}}, \ldots
    $$
    converges and find the limit.
  \item Does the sequence
    $$
    \sqrt{2}, \sqrt{2 \sqrt{2}}, \sqrt{2 \sqrt{2 \sqrt{2}}}, \ldots
    $$
    converge? If so, find the limit.
  }
\end{exercise}

\begin{solution}
  \enum{
  \item To show that this sequence converges, we must show that it is bounded above and monotone increasing.

  First, it is apparent that $y_2 \geq y_1$. Now, given that $y_{k+1} \geq y_k$, then $2 + y_{k+1} \geq 2 + y_k$, and $\sqrt{2 + y_{k+1}} \geq \sqrt{2 + y_k}$. 

  Similarly, it is apparent that $y_1 \leq 4$. Now, given that $y_k \leq 4$, it is easy to see that $y_k + 2 \leq 6$ and so $\sqrt{y_k + 2} \leq 4$.
  Finally, setting $y = \lim y_n$ gives $y=\sqrt{2 + y}\implies y = 2$.
  \item
  The sequence can be rewritten as $2^{\frac{1}{2}}$, $2^\frac{3}{4}$, $2^\frac{7}{8}\ldots$, which is clearly monotone increasing and bounded above. Now $y = \sqrt{2y}\implies y = 2$.
  }
\end{solution}

\begin{exercise}
  \enum{
  \item In Section 1.4 we used the Axiom of Completeness (AoC) to prove the Archimedean Property of $\mathbf{R}$ (Theorem 1.4.2). Show that the Monotone Convergence Theorem can also be used to prove the Archimedean Property without making any use of AoC.
  \item Use the Monotone Convergence Theorem to supply a proof for the Nested Interval Property (Theorem 1.4.1) that doesn't make use of AoC.

  These two results suggest that we could have used the Monotone Convergence Theorem in place of $\mathrm{AoC}$ as our starting axiom for building a proper theory of the real numbers.
  }
\end{exercise}

\begin{solution}
  \enum{
  \item 
  \REPLACE
  Assume, for contradiction, that $\mathbf{N}$ is bounded above. Then the sequence that defines the natural numbers, $x_{n+1} = x_n + 1$, must converge. Taking the limit of both sides gives $0 = 1$, which is clearly false.
  \item 
  Given a sequence of nested intervals, consider the sequence of left-hand endpoints and the sequence of right-hand endpoints. They are monotone increasing and monotone decreasing respectively, and bounded by $b_1$ and $a_1$ respectively. Therefore they must converge, and the Order Limit Theorem, states that $a\leq b$. Now there must exist a $c\in\mathbf{R}$ such that $a\leq c\leq b$, and $c$ is in every interval.
  }
\end{solution}

\begin{exercise}[Calculating Square Roots]
  Let $x_{1}=2$, and define
  $$
  x_{n+1}=\frac{1}{2}\left(x_{n}+\frac{2}{x_{n}}\right)
  $$
  \enum{
  \item Show that $x_{n}^{2}$ is always greater than or equal to 2, and then use this to prove that $x_{n}-x_{n+1} \geq 0$. Conclude that $\lim x_{n}=\sqrt{2}$.
  \item Modify the sequence $\left(x_{n}\right)$ so that it converges to $\sqrt{c}$.
  }
\end{exercise}

\begin{solution}
  \enum{
  \item
  It is trivial to see that $x_1^2 \geq 2$. Now, 
  $$
  \begin{aligned}
  x_n^2 \geq 2
  &\implies (x_n^2-2)^2\geq 0 \\
  &\implies \frac{(x^2_n+2)^2}{4x_n^2} \geq 2 \\
  &\implies \frac{1}{4}\left( x^2_n + 4 + \frac{4}{x^2_n}\right)\geq 2\\
  &\implies x_{n+1}^2\geq 2
  \end{aligned}
  $$
  To see that $x_n - x_{n+1} \geq 0$, $$x_{n+1} - x_{n} = \frac{3}{4}x_n^2 - \frac{1}{x^2_n} - 1 \geq 0$$ for all $x_n^2 \geq 2$. 
  
  Finally, taking the limit of both sides gives $\frac{1}{2}c = \frac{1}{c}\implies c = \sqrt 2$.
  \item 
  $$x_{n+1} = \left(1 - \frac{1}{c}\right)\left( x_n + \frac{1}{x_n(1-\frac{1}{c})}\right)$$
  }
\end{solution}

\begin{exercise}[Arithmetic-Geometric Mean]
  \enum{
  \item Explain why $\sqrt{x y} \leq$ $(x+y) / 2$ for any two positive real numbers $x$ and $y$. (The geometric mean is always less than the arithmetic mean.)
  \item Now let $0 \leq x_{1} \leq y_{1}$ and define
    $$
    x_{n+1}=\sqrt{x_{n} y_{n}} \quad \text { and } \quad y_{n+1}=\frac{x_{n}+y_{n}}{2}
    $$
    Show $\lim x_{n}$ and $\lim y_{n}$ both exist and are equal.
  }

\end{exercise}

\begin{solution}
  \enum{
  \item
  $$
  \begin{aligned}
  \frac{(x-y)^2}{4}\geq 0
  &\implies xy \leq \frac{(x+y)^2}{4} \\
  &\implies \sqrt{xy} \leq \frac{x+y}{2}
  \end{aligned}
  $$
  \item
  First, $(y_n)$ is monotone decreasing and bounded below by $0$. This is because for any $n$, $x_n \leq y_n$ and so $\frac{x_n + y_n}{2}\leq \frac{2y_n}{2}$, and given that $x_n$ and $y_n$ are both greater than $0$, their arithmetic mean must also be greater than $0$. $(x_n)$ is monotone increasing and bounded above by $y_1$. So $y = \lim y_n$ and $x = \lim x_n$ both exist; thus we can take $$y = \frac{x + y}{2}\implies x = y.$$
  }
\end{solution}

\begin{exercise}[Limit Superior]
  \label{ex:lim_sup}
  Let $\left(a_{n}\right)$ be a bounded sequence.

  \enum{
  \item Prove that the sequence defined by $y_{n}=\sup \left\{a_{k}: k \geq n\right\}$ converges.
  \item The limit superior of $\left(a_{n}\right)$, or $\lim \sup a_{n}$, is defined by
    $$
    \limsup a_{n}=\lim y_{n}
    $$
    where $y_{n}$ is the sequence from part (a) of this exercise. Provide a reasonable definition for $\lim \inf a_{n}$ and briefly explain why it always exists for any bounded sequence.
  \item Prove that $\lim \inf a_{n} \leq \lim \sup a_{n}$ for every bounded sequence, and give an example of a sequence for which the inequality is strict.
  \item Show that $\lim \inf a_{n}=\lim \sup a_{n}$ if and only if $\lim a_{n}$ exists. In this case, all three share the same value.
  }
\end{exercise}

\begin{solution}
  \enum{

  \item $(y_n)$ is decreasing and bounded because $a_n$ is bounded.
  \item The limit of the sequence $y_n = \inf \{a_k \mid k\geq n\}$. It exists because the sequence is monotone increasing and bounded.
  \item Every term of the sequence of infimums is less than the corresponding term in the sequence of supremums, so by the Order Limit Theorem, $\lim \inf a_n \leq \lim \sup a_n$. One example of a sequence for which this inequality is strict is $x_n = (-1)^n$.
  \item ($\Rightarrow$) If, for some $n$, both $\sup\{a_k \mid k\geq n\}$ and $\inf\{a_k \mid k\geq n\}$ are within $\epsilon$ of some $c$, then it is clear that $a_k$ is within $\epsilon$ of that same $c$ for $k\geq n$. Thus it has been demonstrated that the limit exists and is $c$.

  ($\Leftarrow$) Fix $\epsilon > 0$. Then there exists some $N$ for which $|x_n - c| < \epsilon/2$ for all $n\geq N$. Now the sequence of infimums, $i_n$, for $n\geq N$ must be greater than or equal to $c - \epsilon/2$ since it is a lower bound on $x_n$; similarly, the sequence of supremums, $s_n$, must be less than or equal to $c + \epsilon/2$. Therefore $|i_n - c| < \epsilon$ and $|s_n - c| < \epsilon$ for all $n\geq N$.
  }
\end{solution}

\begin{exercise}
  For each series, find an explicit formula for the sequence of partial sums and determine if the series converges.
  \enum{
  \item $\sum_{n=1}^{\infty} \frac{1}{2^{n}}$
  \item $\sum_{n=1}^{\infty} \frac{1}{n(n+1)}$
  \item $\sum_{n=1}^{\infty} \log \left(\frac{n+1}{n}\right)$
  }
  (In (c), $\log (x)$ refers to the natural logarithm function from calculus.)
\end{exercise}

\begin{solution}
  \enum{
  \item $S_n = 1 - \frac{1}{2^n}$. This sequence converges to $1$, as can be seen by applying the Archimedean Property.
  \item $S_n = 1 - \frac{1}{n+1}$. Again, this sequence converges to $1$.
  \item $S_n = \log(n+1)$. This sequence diverges.
  }
\end{solution}

\begin{exercise}
  Complete the proof of Theorem 2.4.6 by showing that if the series $\sum_{n=0}^{\infty} 2^{n} b_{2^{n}}$ diverges, then so does $\sum_{n=1}^{\infty} b_{n}$. Example $2.4.5$ may be a useful reference.
\end{exercise}

\begin{solution}
  First, divide the partial sum $t_k$ by $2$. This partial sum is still unbounded for all $k$ as if it were bounded by $M$, then $t_k$ would be bounded by $2M$. So the new partial sum is then $$t_k = \frac{b_1}{2} + b_2 + 2b_4 + \ldots + 2^{k-1}b_{2^k}.$$ Now consider 
  $$
  \begin{aligned}
  s_{2^k} 
  &= (b_1) + (b_2) + (b_3 + b_4) + \ldots + (b_{2^{k-1} + 1} + \ldots + b_{2k})\\
  &\geq \frac{b_1}{2} + b_2 + 2b_4 + \ldots + 2^{k-1}b_{2^k}\\
  &= t_k,\\
  \end{aligned}
  $$
  which is unbounded. Therefore the partial sums must also be unbounded, and the sum diverges.

  This completes the proof. We have proved that $A'$ implies $B'$; this is the same as saying that $B$ implies $A$ since there is no way for $B$ to be true without $A$ being true.
\end{solution}

\begin{exercise}[Infinite Products]
  A close relative of infinite series is the infinite product
  $$
  \prod_{n=1}^{\infty} b_{n}=b_{1} b_{2} b_{3} \cdots
  $$
  which is understood in terms of its sequence of partial products
  $$
  p_{m}=\prod_{n=1}^{m} b_{n}=b_{1} b_{2} b_{3} \cdots b_{m}
  $$

  Consider the special class of infinite products of the form
  $$
  \prod_{n=1}^{\infty}\left(1+a_{n}\right)=\left(1+a_{1}\right)\left(1+a_{2}\right)\left(1+a_{3}\right) \cdots, \quad \text { where } a_{n} \geq 0
  $$
  \enum{
  \item Find an explicit formula for the sequence of partial products in the case where $a_{n}=1 / n$ and decide whether the sequence converges. Write out the first few terms in the sequence of partial products in the case where $a_{n}=1 / n^{2}$ and make a conjecture about the convergence of this sequence.
  \item Show, in general, that the sequence of partial products converges if and only if $\sum_{n=1}^{\infty} a_{n}$ converges. (The inequality $1+x \leq 3^{x}$ for positive $x$ will be useful in one direction.)
  }
\end{exercise}

\begin{solution}
  \enum{
  \item This is a telescoping product, most of the terms cancel
    $$
    p_m = \prod_{n=1}^m (1 + 1/n) = \prod_{n=1}^m \frac{n+1}{n} = \frac 21 \cdot \frac 32 \cdot \frac 42 \cdots \frac {m+1}{m} = m+1
    $$
    therefore $(p_m)$ diverges.

    In the cast $a_n = 1/n^2$ we get
    $$
    \prod_{n=1}^\infty (1+ 1/n^2) = \prod_{n=1}^\infty \frac{1 + n^2}{n^2} = \frac 21 \cdot \frac 54 \cdot \frac{10}{9} \cdots
    $$ 
    The growth seems slower, I conjecture it converges now.
  \item Using the inequality suggested we have $1 + a_n \le 3^{a_n}$ letting $s_m = a_1 + \dots + a_m$ we get
    $$
    p_m = (1+a_1)\cdots(1+a_m) \le 3^{a_1}3^{a_2}\cdots 3^{a_m} = 3^{s_m}
    $$
    Now if $s_m$ converges it is bounded by some $M$ meaning $p_m$ is bounded by $3^M$. and because $a_n \ge 0$ the partial products $p_m$ are increasing, so they converge by the MCT. This shows $s_m$ converging implies $p_m$ converges.

    For the other direction suppose $p_m \to p$. Distributing inside the products gives $p_2 = a_1 + a_2 + 1 + a_1a_2 > s_2$ and in general $p_m > s_m$ implying that if $p_m$ is bounded then $s_n$ is bounded aswell. This completes the proof.

    \textbf{Summary}: Convergence is if and only if because $s_m \le p_m \le 3^{s_m}$.

    (By the way the inequality $1 + x \le 3^x$ can be derived from $\log(1 + x) \le x$ implying $1 + x \le e^x$, I assume abbott rounded up to $3$.)
  }
\end{solution}