\section{Some Preliminaries}

\begin{exercise}
  \enum{
  \item Prove that $\sqrt 3$ is irrational. Does a similar similar argument work to show $\sqrt 6$ is irrational?
  \item Where does the proof break down if we try to prove $\sqrt 4$ is irrational?
  }
\end{exercise}

\begin{solution}
  \enum{
  \item Assume, for contradiction, that there exists two integers $p$ and $q$ such that $p/q = \sqrt{3}$, and $p/q$ is in lowest terms. Then $(p/q)^2 = 3$, and $p^2 = 3q^2$. Thus $p$ is divisible by $3$, and we can write $p = 3r$. Rearranging, however, we get $q^2 = 3r^2$, which implies that $q$ is divisible by $3$ as well and contradicts our assumption that $p/q$ was in lowest terms. The same proof holds for $\sqrt{6}$.
  \item $p^2$ is divisible by $4$ $\implies$ $p$ is divisible by $4$ does not hold.
  }
\end{solution}

\begin{exercise}
  Show that there is no rational number satisfying $2^r = 3$.
\end{exercise}

\begin{solution}
  If $2^r = 3$ and $r = p/q$ where $p$ and $q$ are integers, then $2^{p/q} = 3$ or $2^p = 3^q$. This is impossible, so $r$ must not be a rational number.
\end{solution}

\begin{exercise}
  Decide which of the following represent true statements about the nature of sets. For any that are false, provide a specific example where the statement in question does not hold.
  \enum{
  \item If $A_{1} \supseteq A_{2} \supseteq A_{3} \supseteq A_{4} \cdots$ are all sets containing an infinite number of elements, then the intersection $\bigcap_{n=1}^{\infty} A_{n}$ is infinite as well.
  \item If $A_{1} \supseteq A_{2} \supseteq A_{3} \supseteq A_{4} \cdots$ are all finite, nonempty sets of real numbers, then the intersection $\bigcap_{n=1}^{\infty} A_{n}$ is finite and nonempty.
  \item $A \cap(B \cup C)=(A \cap B) \cup C$.
  \item $A \cap(B \cap C)=(A \cap B) \cap C$.
  \item $A \cap(B \cup C)=(A \cap B) \cup(A \cap C)$.
  }
\end{exercise}

\begin{solution}
  \enum{
  \item False. For example, consider $\bigcap_{n=0}^{\infty} A_n$ where $A_n = \{k2^n \mid k \in \mathbf{Z}^+\}$, where every set contains the all multiples of $2^n$ and is therefore infinite, but the only common element is $0$.
  \item True. $\bigcap_{n=1}^{\infty} A_n$ must be a subset of every $A_n$. However, $A_n$ is finite, and an infinite set cannot be the subset of a finite set.
  \item False. The set on the right includes all of $C$, whereas the set on the left includes only $A \cap C$. If $A \neq C$, then this falls apart.
  \item True. An element is in all three sets if and only if it is in both the left and the right set.
  \item True. If $x \in A \cap (B \cup C)$, then $x \in A \cap B$ or $x \in A \cap C$.
  }
\end{solution}

\begin{exercise}
  Produce an infinite collection of sets $A_{1}, A_{2}, A_{3}, \ldots$ with the property that every $A_{i}$ has an infinite number of elements, $A_{i} \cap A_{j}=\emptyset$ for all $i \neq j$, and $\bigcup_{i=1}^{\infty} A_{i}=\mathbf{N}$
\end{exercise}

\begin{solution}
  Consider $A_n = \{k2^{n-1} \mid k \mathrm{\:is\:odd}\}$, all of which are obviously infinite. Each element of $A_n$ is also some odd multiple of $2^{n-1}$. Thus any element $x \in A_n$ cannot be in $A_m$ for all $m < n$ as $x$ can be expressed as an even multiple of $2^{m-1}$, so $A_n$ is disjoint. Any $k \in \mathbf N$ can be expressed as $2^{a}b$, where $2^a$ is the highest power of $2$ that $k$ divides. This implies that $b$ is odd, so $k \in A_{a+1}$.
\end{solution}

\begin{exercise}[De Morgan's Laws]
  \label{demorgan} 
  Let $A$ and $B$ be subsets of $\mathbf R$.
  \enum{
  \item If $x \in(A \cap B)^{c}$, explain why $x \in A^{c} \cup B^{c}$. This shows that $(A \cap B)^{c} \subseteq$ $A^{c} \cup B^{c}$.
  \item Prove the reverse inclusion $(A \cap B)^{c} \supseteq A^{c} \cup B^{c}$, and conclude that $(A \cap B)^{c}=A^{c} \cup B^{c}$.
  \item Show $(A \cup B)^{c}=A^{c} \cap B^{c}$ by demonstrating inclusion both ways.
  }
\end{exercise}
  
\begin{solution}
  \enum{
  \item $x\in (A\cap B)^c\iff x\notin A\cap B$, which is to say that either $x\notin A$ or $x\notin B$. Therefore $x\in A^c\cup B^c$, which means that $(A\cap B)^c \subseteq A^c\cup B^c$.
  \item For the reverse, the same proof holds, as all statements in (a) are true in both directions.
  \item $x\in (A\cup B)^c\iff x\notin A\cup B\iff x$ is not in $A$ or $B\iff x\in A^c\cap B^c$.
  }
\end{solution}

\begin{exercise}
  \enum{
  \item Verify the triangle inequality in the special case where $a$ and $b$ have the same sign.
  \item Find an efficient proof for all the cases at once by first demonstrating $(a+b)^{2} \leq(|a|+|b|)^{2}$.
  \item Prove $|a-b| \leq|a-c|+|c-d|+|d-b|$ for all $a, b, c$, and $d$.
  \item Prove $\| a|-| b|| \leq|a-b|$. (The unremarkable identity $a=a-b+b$ may be useful.)
  }
\end{exercise}

\begin{solution}
  \enum{
  \item When $a$ and $b$ have the same sign, $|a+b|=|a|+|b|$ and so the triangle inequality is true.
  \item $(|a|+|b|)^2=a^2+2|a||b|+b^2\geq a^2+2ab+b^2=(a+b)^2\implies |a+b|\leq |a|+|b|$ as squaring a value and taking its positive square root is equivalent to taking its absolute value.
  \item By the triangle inequality, $|a-b| = |(a-c)+(c-d)+(d-b)|\leq |a-c|+|(c-d)+(d-b)|\leq |a-c|+|c-d|+|d-b|$.
  \item $(|a|-|b|)^2=a^2-2|a||b|+b^2\leq a^2-2ab+b^2=(a-b)^2\implies |a-b|\geq ||a|-|b||$.
  }
\end{solution}

\begin{exercise}
  Given a function $f$ and a subset $A$ of its domain, let $f(A)$ represent the range of $f$ over the set $A$; that is, $f(A)=\{f(x): x \in A\}$.
  \enum{
  \item Let $f(x)=x^{2} .$ If $A=[0,2]$ (the closed interval $\left.\{x \in \mathbf{R}: 0 \leq x \leq 2\}\right)$ and $B=[1,4]$, find $f(A)$ and $f(B) .$ Does $f(A \cap B)=f(A) \cap f(B)$ in this case? Does $f(A \cup B)=f(A) \cup f(B) ?$
  \item Find two sets $A$ and $B$ for which $f(A \cap B) \neq f(A) \cap f(B)$.
  \item  Show that, for an arbitrary function $g: \mathbf{R} \rightarrow \mathbf{R}$, it is always true that $g(A \cap B) \subseteq g(A) \cap g(B)$ for all sets $A, B \subseteq \mathbf{R}$.
  \item Form and prove a conjecture about the relationship between $g(A \cup B)$ and $g(A) \cup g(B)$ for an arbitrary function $g$.
  }
\end{exercise}
  
\begin{solution}
  \enum{
  \item $f(A)=[0,4]$, and $f(B)=[1,16]$. In this case, $$f(A\cap B)=f([1,2])=[1,4]=f(A)\cap f(B),$$ and $$f(A\cup B)=f([0,4])=[0,16]=f(A)\cup f(B).$$
  \item Let $A=[0,1]$ and $B=[-1,0]$. Then $f(A\cap B)=f({0})={0}$, but $f(A)\cap f(B)=[0,1]$.
  \item Let $y\in g(A\cap B)$ be an arbitrary element, and $x\in A\cap B$ the element such that $f(x) = y$. Then, since $x$ is in both $A$ and $B$, $f(x)\in g(A)\cap g(B)$.
  \item Conjecture: for any arbitrary function $g: \mathbf{R} \rightarrow \mathbf{R}$, it is always true that $g(A\cup B) = g(A)\cup g(B)$. To prove this, we show inclusion both ways. For any $y\in g(A\cup B)$, there exists $x\in A\cup B$ such that $f(x)=y$. Therefore, $y\in g(A)\cup g(B)$. For the reverse, if $y\in g(A)\cup g(B)$, then $x\in A\cup B$, and so $y\in g(A\cup B)$.
  }
\end{solution}

\begin{exercise}
  Here are two important definitions related to a function $f:$ $A \rightarrow B .$ The function $f$ is \emph{one-to-one} $(1-1)$ if $a_{1} \neq a_{2}$ in $A$ implies that $f\left(a_{1}\right) \neq$ $f\left(a_{2}\right)$ in $B$. The function $f$ is \emph{onto} if, given any $b \in B$, it is possible to find an element $a \in A$ for which $f(a)=b$
  Give an example of each or state that the request is impossible:
  \enum{
  \item $f: \mathbf{N} \rightarrow \mathbf{N}$ that is $1-1$ but not onto.
  \item $f: \mathbf{N} \rightarrow \mathbf{N}$ that is onto but not $1-1$.
  \item $f: \mathbf{N} \rightarrow \mathbf{Z}$ that is $1-1$ and onto.
  }
\end{exercise}
  
\begin{solution}
  \enum{
  \item $f(n) = n + 1$ is $1-1$, but not onto.
  \item $f(n) = \lfloor \frac{n}{2} \rfloor + 1$ is onto, but not $1-1$.
  \item To construct a bijection from $\mathbf{N}$ to $\mathbf{Z}$, define 
  $$
  f(n) = \begin{cases}
    \frac{n + 1}{2} &\text{if } n \text{ is odd}\\
    -\frac{n}{2} &\text{if } n \text{ is even}\\
  \end{cases}
  $$
  Then the odd natural numbers map onto the natural numbers, whereas the even natural numbers map onto the negative integers and zero. This function is a bijection since it maps every natural number maps to a unique integer and every integer is mapped onto by some natural number.
  }
\end{solution}

\begin{exercise}
  Given a function $f: D \rightarrow \mathbf{R}$ and a subset $B \subseteq \mathbf{R}$, let $f^{-1}(B)$ be the set of all points from the domain $D$ that get mapped into $B ;$ that is, $f^{-1}(B)=\{x \in D: f(x) \in B\} .$ This set is called the \emph{preimage} of $B$.
  \enum{
  \item Let $f(x)=x^{2} .$ If $A$ is the closed interval $[0,4]$ and $B$ is the closed interval $[-1,1]$, find $f^{-1}(A)$ and $f^{-1}(B)$. Does $f^{-1}(A \cap B)=f^{-1}(A) \cap f^{-1}(B)$ in this case? Does $f^{-1}(A \cup B)=f^{-1}(A) \cup f^{-1}(B) ?$
  \item The good behavior of preimages demonstrated in (a) is completely general. Show that for an arbitrary function $g: \mathbf{R} \rightarrow \mathbf{R}$, it is always true that $g^{-1}(A \cap B)=g^{-1}(A) \cap g^{-1}(B)$ and $g^{-1}(A \cup B)=g^{-1}(A) \cup g^{-1}(B)$ for all sets $A, B \subseteq \mathbf{R}$.
  }
\end{exercise}

\begin{solution}
  \enum{
  \item $f^{-1}(A)=[-2,2]$, $f^{-1}(B)=[-1,1]$. $f^{-1}(A\cap B)=[-1,1]=f^{-1}(A)\cap f^{-1}(B)$. $f^{-1}(A\cup B)=[-2,2]=f^{-1}(A)\cup f^{-1}(B)$.
  \item $x\in g^{-1}(A\cap B)\iff y\in A\cap B\text{ where } g(x)=y\iff y$ is in $A$ and $B\iff x\in g^{-1}(A)\cap g^{-1}(B)$.

  Similarly, $x\in g^{-1}(A\cup B)\iff y\in A\cup B\text { where } g(x)=y\iff y$ is in $A$ or $B\iff x\in g^{-1}(A)\cup g^{-1}(B)$.
  }
\end{solution}


\begin{exercise}
  Decide which of the following are true statements. Provide a short justification for those that are valid and a counterexample for those that are not:
  \enum{
  \item Two real numbers satisfy $a<b$ if and only if $a<b+\epsilon$ for every $\epsilon>0$.
  \item Two real numbers satisfy $a<b$ if $a<b+\epsilon$ for every $\epsilon>0$.
  \item Two real numbers satisfy $a \leq b$ if and only if $a<b+\epsilon$ for every $\epsilon>0$.
  }
\end{exercise}


\begin{solution}
  \enum{
  \item False. If $a=b$, then $a<b+\epsilon$ for every $\epsilon>0$.
  \item False. Same as part (a).
  \item True. If $a>b$, then there would be some $\epsilon\leq a-b$ for which $a<b+\epsilon$ does not hold, so $a<b+\epsilon\implies a\leq b$. Conversely, it is easy to see that $a\leq b\implies a<b+\epsilon$ for every $\epsilon>0$.
  }
\end{solution}

\begin{exercise}
  Form the logical negation of each claim. One trivial way to do this is to simply add ``It is not the case that...'' in front of each assertion. To make this interesting, fashion the negation into a positive statement that avoids using the word ``not'' altogether. In each case, make an intuitive guess as to whether the claim or its negation is the true statement.
  \enum{
  \item For all real numbers satisfying $a<b$, there exists an $n \in \mathbf{N}$ such that $a+1 / n<b$.
  \item There exists a real number $x>0$ such that $x<1 / n$ for all $n \in \mathbf{N}$.
  \item Between every two distinct real numbers there is a rational number.
  }
\end{exercise}
  
\begin{solution}
  \enum{
  \item There exist real numbers satisfying $a<b$ such that $a+1/n\geq b$ for all $n\in\mathbf{N}$. (Original)
  \item For all real numbers $x>0$, there exists $n\in\mathbf{N}$ such that $x \geq 1/n$. (Negation)
  \item There exist two distinct real numbers without a rational number between them. (Original)
  }
\end{solution}


\begin{exercise}
  Let $y_{1}=6$, and for each $n \in \mathbf{N}$ define $y_{n+1}=\left(2 y_{n}-6\right) / 3$.
  \enum{
  \item Use induction to prove that the sequence satisfies $y_{n}>-6$ for all $n \in \mathbf{N}$.
  \item Use another induction argument to show the sequence $\left(y_{1}, y_{2}, y_{3}, \ldots\right)$ is decreasing.
  }
\end{exercise}

\begin{solution}
  First, $y_1>-6$.
  Next, if $y_k>-6$, then $2y_k>-12$ and $(2y_k-6)/3>-6$. Therefore $y_k>-6$ implies $y_{k+1}>-6$, and so $y_n>-6$ for all $n\in\mathbf{N}$.
\end{solution}

\begin{exercise}
  For this exercise, assume Exercise \ref{demorgan} has been successfully completed.
  \enum{
  \item Show how induction can be used to conclude that
    $$
    \left(A_{1} \cup A_{2} \cup \cdots \cup A_{n}\right)^{c}=A_{1}^{c} \cap A_{2}^{c} \cap \cdots \cap A_{n}^{c}
    $$
    for any finite $n \in \mathbf{N}$.
  \item It is tempting to appeal to induction to conclude
    $$
    \left(\bigcup_{i=1}^{\infty} A_{i}\right)^{c}=\bigcap_{i=1}^{\infty} A_{i}^{c}
    $$
    but induction does not apply here. Induction is used to prove that a particular statement holds for every value of $n \in \mathbf{N}$, but this does not imply the validity of the infinite case. To illustrate this point, find an example of a collection of sets $B_{1}, B_{2}, B_{3}, \ldots$ where $\bigcap_{i=1}^{n} B_{i} \neq \emptyset$ is true for every $n \in \mathbf{N}$, but $\bigcap_{i=1}^{\infty} B_{i} \neq \emptyset$ fails.
  \item Nevertheless, the infinite version of De Morgan's Law stated in (b) is a valid statement. Provide a proof that does not use induction.
  }
\end{exercise}

\begin{solution}
  \enum{
  \item It is easy to show that $\left(\bigcup_{i=1}^{n} A_i \right)^c=A_i^c$ when $n=1$.

  It is given that $$\left(\bigcup^n_{i=1}A_i\right)^c = \bigcap_{i=1}^{n} A_{i}^{c}.$$ Now, by De Morgan's law, $$(A_1\cup\ldots\cup A_{n+1})^c = (A_1\cup\ldots\cup A_{n})^c \cap A_{n+1}^c.$$ This, however, is equal to $$A_1^c\cap\ldots\cap A_{n+1}^c,$$ and so De Morgan's law holding for $n$ unions implies that it holds for $n+1$ unions. Therefore it holds for $n$ unions for all $n\in\mathbf{N}$.
  \item Consider the collection of sets $S_n = \{k2^n \mid k\in\mathbf{N}\}$.

  For any finite $n$, $\bigcap_{i=1}^{n} S_i = S_n$. However, $\bigcap_{i=1}^{\infty} S_i=\emptyset$.
  \item Show that $$\left(\bigcup_{i=1}^{\infty} A_{i}\right)^{c}=\bigcap_{i=1}^{\infty} A_{i}^{c}.$$ 

  $x$ belongs to the set on the left $\iff x \notin \bigcup_{i=1}^{\infty} A_i\iff x\notin A_i$ for all $i\in\mathbf{N}\iff x\in A_i^c$ for all $i\in\mathbf{N}\iff x$ belongs to the set on the right.
  }
\end{solution}