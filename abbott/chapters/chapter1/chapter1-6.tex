\section{Cantor's theorem}

\begin{exercise}
  Show that $(0,1)$ is uncountable if and only if $\mathbf{R}$ is uncountable.
\end{exercise}

\begin{solution}
  $$f(x)=\tan{\left(y+\frac{\pi}{2}=\pi x \right)}$$ gives a bijection from $(0, 1)$ to $\mathbf{R}$. 
\end{solution}

\begin{exercise}
  Let $f : \mathbf{N} \to \mathbf{R}$ be a way to list every real number (hence show $\mathbf R$ is countable).

  Define a new number $x$ with digits $b_1b_2\ldots$ given by
  $$
  b_{n}= \begin{cases}2 & \text { if } a_{n n} \neq 2 \\ 3 & \text { if } a_{n n}=2\end{cases}
  $$

  \enum{
  \item Explain why the real number $x=. b_{1} b_{2} b_{3} b_{4} \ldots$ cannot be $f(1)$.
  \item Now, explain why $x \neq f(2)$, and in general why $x \neq f(n)$ for any $n \in \mathbf{N}$.
  \item Point out the contradiction that arises from these observations and conclude that $(0,1)$ is uncountable.
  }
\end{exercise}

\begin{solution}
  \enum{
  \item $a_11\neq b_1$.
  \item $a_nn\neq b_n$.
  \item We have found a new number in $(0, 1)$; however, this contradicts our assumption that every real number was in our table and corresponded to a natural number.
  }
\end{solution}

\begin{exercise}
  Supply rebuttals to the following complaints about the proof of Theorem 1.6.1.
  \enum{
  \item Every rational number has a decimal expansion, so we could apply this same argument to show that the set of rational numbers between 0 and 1 is uncountable. However, because we know that any subset of $\mathbf{Q}$ must be countable, the proof of Theorem 1.6.1 must be flawed.
  \item Some numbers have two different decimal representations. Specifically, any decimal expansion that terminates can also be written with repeating 9's. For instance, $1 / 2$ can be written as $.5$ or as $.4999 \ldots$ Doesn't this cause some problems?
  }
\end{exercise}

\begin{solution}
  \enum{
  \item Every rational number having a decimal expansion does not imply every decimal expansion having a rational representation. In fact, because the constructed number has no repeating decimal pattern, it will not be in the set of rational numbers and $\mathbf{Q}$ remains countable.
  \item If we always assume the infinite representation, numbers with terminating decimal expansions are no different from any other real number.
  }
\end{solution}

\begin{exercise}
  Let $S$ be the set consisting of all sequences of 0 's and 1 's. Observe that $S$ is not a particular sequence, but rather a large set whose elements are sequences; namely,
  $$
  S=\left\{\left(a_{1}, a_{2}, a_{3}, \ldots\right): a_{n}=0 \text { or } 1\right\}
  $$
  As an example, the sequence $(1,0,1,0,1,0,1,0, \ldots)$ is an element of $S$, as is the sequence $(1,1,1,1,1,1, \ldots)$.
  Give a rigorous argument showing that $S$ is uncountable.
\end{exercise}

\begin{solution}
  Organize all the sequences in a table indexed by $n\in\mathbf{N}$. Then for $a_nn$, the $n$th number of the $n$th sequence, add the opposite to our new sequence. This is a new sequence in $S$ that does not have an index.
\end{solution}

\begin{exercise}
  \enum{
  \item Let $A=\{a, b, c\}$. List the eight elements of $P(A)$. (Do not forget that $\emptyset$ is considered to be a subset of every set.)
  \item If $A$ is finite with $n$ elements, show that $P(A)$ has $2^{n}$ elements.
  }
\end{exercise}

\begin{solution}
  \enum{
  \item
  $\emptyset$

  $\{a\}$, $\{b\}$, $\{c\}$

  $\{a, b\}$, $\{b, c\}$, $\{a, c\}$

  $\{a, b, c\}$
  \item
  When constructing a subset, we have two options for each element: include it or leave it out, which corresponds to $2^n$ ways to construct a subset from a set of size $n$.
  }
\end{solution}

\begin{exercise}
  \enum{
  \item Using the particular set $A=\{a, b, c\}$, exhibit two different $1-1$ mappings from $A$ into $P(A)$.
  \item Letting $C=\{1,2,3,4\}$, produce an example of a $1-1$ map $g: C \rightarrow P(C)$.
  \item Explain why, in parts (a) and (b), it is impossible to construct mappings that are onto.
  }
\end{exercise}

\begin{solution}
  \enum{
  \item $f(x) = \{x\}$, $f(x)=\{x, y\}$ where $y$ is the next element in the set.
  \item $f(x) = \{x\}$.
  \item Because the $P(A)$ has more elements than $A$ for the examples.
  }
\end{solution}

\begin{exercise}
  \SKIP
\end{exercise}

\begin{exercise}
  \enum{
  \item First, show that the case $a^\prime\in B$ leads to a contradiction.
  \item Now, finish the argument by showing that the case $a^\prime\notin B$ is equally unacceptable.
  }
\end{exercise}

\begin{solution}
  \enum{
  \item $a^\prime\in B$ does not work, because $f(a^\prime)=B$, and by definition we would have constructed $B$ to not include $a^\prime$ if that were the case.
  \item $a^\prime\notin B$ does not work, because then we would have constructed $B$ to include $a^\prime$. 
  }
\end{solution}

\begin{exercise}
  Using the various tools and techniques developed in the last two sections (including the exercises from Section 1.5), give a compelling argument showing that $P(\mathbf{N}) \sim \mathbf{R}$.
\end{exercise}

\begin{solution}
  To show that $P(\mathbf{N})\sim \mathbf{R}$, first consider $(0, 1)\sim \mathbf{R}$. Note that our choice of base 10 is arbitrary; we can just as easily represent decimal expansions in binary. The bijection is constructed as follows: if the $n$th digit of the binary expansion of real number $x$ is $1$, then include $n$; if not, do not include $n$. This is an injection from $\mathbf{R}$ to $P(\mathbf{N})$ because if $a\neq b$, then there is an $n\in\mathbf{N}$ such that $a_n\neq b_n$. Therefore $f(a)\neq f(b)$, as only one of them includes $n$. It is onto because all subsets of $\mathbf{N}$ can be generated by determining which elements to leave out.
\end{solution}

\begin{exercise}
  As a final exercise, answer each of the following by establishing a $1-1$ correspondence with a set of known cardinality.
  \enum{
  \item Is the set of all functions from $\{0,1\}$ to $\mathbf{N}$ countable or uncountable?
  \item Is the set of all functions from $\mathbf{N}$ to $\{0,1\}$ countable or uncountable?
  \item Given a set $B$, a subset $\mathcal{A}$ of $P(B)$ is called an antichain if no element of $\mathcal{A}$ is a subset of any other element of $\mathcal{A} .$ Does $P(\mathbf{N})$ contain an uncountable antichain?
  }
\end{exercise}

\begin{solution}
  \enum{
  \item This is the same as $\mathbf{N^2}$, which is countable.
  \item This is the same proof from 1.6.4. It is uncountable.
  \item Yes. For example, construct an antichain $A$ containing all subsets $A_k \subseteq \mathbf{N}$ such $A_k$ contains either $2n$ or $2n-1$ but not both for any $n\in\mathbf{N}$. This is equivalent to the set of infinite sequences of $0$s and $1$s, with $0$ denoting the even number picked and $1$ denoting the odd number picked for each $n$; however, this set known to be uncountable, so $A$ must also be uncountable.
  }
\end{solution}